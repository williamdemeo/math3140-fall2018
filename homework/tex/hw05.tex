\documentclass[12pt,reqno]{amsart}
\usepackage[top=1.5cm, left=1.5cm,right=1.5cm,bottom=1.5cm]{geometry}
\renewcommand{\baselinestretch}{1.2}
\usepackage{amsmath}
\usepackage{amssymb}
\usepackage{color,hyperref,enumerate,multicol}
\definecolor{darkblue}{rgb}{0.0,0.0,0.3}
\hypersetup{colorlinks,breaklinks,
            linkcolor=darkblue,urlcolor=darkblue,
            anchorcolor=darkblue,citecolor=darkblue}
\usepackage{tikz}
\usepackage{scalefnt}
\usepackage{algorithm}
\usepackage{algorithmic}
\pagestyle{empty}
\newcommand{\N}{\ensuremath{\mathbb{N}}}
\newcommand{\Z}{\ensuremath{\mathbb{Z}}}
\newcommand{\R}{\ensuremath{\mathbb{R}}}
\newcommand{\meet}{\ensuremath{\wedge}}
\newcommand{\Meet}{\ensuremath{\bigwedge}}
\newcommand{\join}{\ensuremath{\vee}}
\renewcommand{\emptyset}{\ensuremath{\varnothing}}
\renewcommand{\subset}{\ensuremath{\subsetneq}}
\newcommand{\boldemph}{\emph}
\newcommand{\lcm}{\operatorname{lcm}}

\begin{document}
\thispagestyle{empty}

\noindent \textbf{Math 3140 -- Homework 5} \hfill {\bf Due date:} Monday, 10/1\\[4pt]
\noindent {\bf Required Exercises.  Chapter 4:  6, 12, 13, 28, 30, 35}.

\begin{enumerate}

%% 6 %%%%%%%%%%%%%%%%%%%%%%%%%%%%%%%%%%%%%%%%%%%%%%%%
\item[{\bf 6.}]
Find the order of every element in the symmetry group of the square,
$D_4$.

\bigskip

%% 12 %%%%%%%%%%%%%%%%%%%%%%%%%%%%%%%%%%%%%%%%%%%%%%%%
\item[{\bf 12.}]
Find a cyclic group with exactly one generator.  Can you find cyclic
groups with exactly two generators?  Four generators?  How about $n$
generators?

\bigskip

%% 13 %%%%%%%%%%%%%%%%%%%%%%%%%%%%%%%%%%%%%%%%%%%%%%%%
\item[{\bf 13.}]
For $n \leq 20$, which groups $U(n)$ are cyclic?  Make a conjecture as
to what is true in general.  Can you prove your conjecture?  

\bigskip

%% 28 %%%%%%%%%%%%%%%%%%%%%%%%%%%%%%%%%%%%%%%%%%%%%%%%
\item[{\bf 28.}]
If $a$ is an element of a group, what is a generator for the
subgroup $\langle a^m \rangle  \cap  \langle a^n \rangle $?
 

\bigskip

%% 30 %%%%%%%%%%%%%%%%%%%%%%%%%%%%%%%%%%%%%%%%%%%%%%%%
\item[{\bf 30.}]
Suppose $a$ and $b$ are elements of a group. Prove that if $|a|
= m$ and $|b| = n$ with $\gcd(m,n) = 1$, then $\langle a \rangle  \cap
\langle b \rangle  = \{ e \}$. 
 
\bigskip

%% 35 %%%%%%%%%%%%%%%%%%%%%%%%%%%%%%%%%%%%%%%%%%%%%%%%
\item[{\bf 35.}]
Prove that the subgroups of ${\mathbb Z}$ are exactly $n{\mathbb Z}$ for $n
= 0, 1, 2, \ldots$. 

\end{enumerate}

\bigskip

\bigskip

\noindent {\bf Suggested Exercises}.  Chapter 4:  38; plus Exercise 47 (stated below).

\medskip

\begin{enumerate}

%% 38 %%%%%%%%%%%%%%%%%%%%%%%%%%%%%%%%%%%%%%%%%%%%%%%%
\item[{\bf 38.}]
Prove that the order of an element in a cyclic group must divide
the order of the  group. 

\bigskip

%%%%%%%%%%%%%%%%%%%%%%%%%%%%%%%%%%%%%%%%%%%%%%%%%%
\item[{\bf 47.}]
Explain why the Hasse diagram below cannot be the entire subgroup lattice of a group.\\
(This exercise does not appear in the textbook.)

\bigskip
\newcommand{\dotsize}{1pt}
%%
%% To create nodes of lattices in a uniform and consistent way, we define
\tikzstyle{lat} = [circle,draw,inner sep=\dotsize]
%% To put a lattice node named "mynode" at the point (x,y)=(1,2) in the figure,
%% put the following inside your tikzpicture block:
%%     \node[lat] (mynode) at (1,2) {};

% To scale all diagrams uniformly, change this setting:
\newcommand{\figscale}{1.2}
\begin{center}
{  \scalefont{0.8}
  \begin{tikzpicture}[scale=\figscale]
    \draw (0.2,-.2) node {$e$};
    \draw (0.2,3.2) node {$G$};
    \draw (-1.2,1) node {$H$};
    \draw (0.2,2.2) node {$J$};
    \draw (1.2,1) node {$K$};
  \node[lat] (bot) at (0,0) {};
\node[lat] (top) at (0,3) {};
\node[lat] (maxl) at (0,2) {};
\node[lat] (a) at (-1,1) {};
\node[lat] (c) at (1,1) {};
\draw[semithick] (bot) -- (a) -- (maxl) -- (c) -- (bot);
\draw[semithick] (maxl) -- (top);
\end{tikzpicture}
}
\end{center}
That is, $H$, $J$, and $K$ are supposed to be subgroups of $G$ such that $H\leq J$ and $K\leq J$; 
moreover, there is supposed to be exactly one maximal subgroup of $G$, namely $J$; finally, $H \cap K = \langle e \rangle$.
({\it Hint:} Does there exist an element of $G$ that does not belong to $J$?)
\end{enumerate}


\end{document}
